\documentclass[10pt,a4paper]{report}
\usepackage[latin1]{inputenc}
\usepackage{amsmath}
\usepackage{amsfonts}
\usepackage{amssymb}
\usepackage{makeidx}
\usepackage{graphicx}
\usepackage[left=2.00cm, right=2.00cm, top=2.00cm, bottom=2.00cm]{geometry}


\usepackage{listings}
\usepackage{color}

\definecolor{dkgreen}{rgb}{0,0.6,0}
\definecolor{gray}{rgb}{0.5,0.5,0.5}
\definecolor{mauve}{rgb}{0.58,0,0.82}

\lstset{frame=tb,
	language=Java,
	aboveskip=3mm,
	belowskip=3mm,
	showstringspaces=false,
	columns=flexible,
	basicstyle={\small\ttfamily},
	numbers=none,
	numberstyle=\tiny\color{gray},
	keywordstyle=\color{blue},
	commentstyle=\color{dkgreen},
	stringstyle=\color{mauve},
	breaklines=true,
	breakatwhitespace=true,
	tabsize=4
}




\title{A4 MiniTest Answers}
\author{Joel Busch, Jakob Meier, Pascal Oberholzer}

\begin{document}
	\maketitle
	
	\section*{1. (Sensor Framework)}
		\subsection*{A) Code Snippets}
			\subsubsection*{a) List available sensors of a device}
				\begin{lstlisting}
					//edit list_view
					ListView list_view = (ListView) findViewById(R.id.sensor_list);
					list_view.setOnItemClickListener(this);
					
					//get all sensors
					SensorManager mgr = (SensorManager) getSystemService(SENSOR_SERVICE);
					List<Sensor> sensors = mgr.getSensorList(Sensor.TYPE_ALL);
					
					SensorArrayAdapter<Sensor> array_adapter = new SensorArrayAdapter<Sensor>(this,
					android.R.layout.simple_list_item_1,
					sensors);
					
					list_view.setAdapter(array_adapter);
				\end{lstlisting}

			\subsubsection*{b) Retrieve the value range of a specific Sensor}
				\begin{lstlisting}
					Sensor item;
					float maxRange = item.getMaximumRange();
				\end{lstlisting}

			\subsubsection*{c) Register for monitoring Accelerometer sensor changes at the maximum available rate}
				\begin{lstlisting}
					//initialize a SensorManager
					SensorManager sm = (SensorManager)getSystemService(SENSOR_SERVICE);
					Sensor sensor;
					
					//initialize listener
					SensorEventListener listener;
					listener = new ... ;
					
					//get ACCELERATION sensor
					sensor = sm.getDefaultSensor(Sensor.TYPE_ACCELEROMETER);
					
					//register Listener with fastest available delay
					sm.registerListener(listener, sensor, SensorManager.SENSOR_DELAY_FASTEST);
				\end{lstlisting}
				
				
		\subsection*{B) Search the Mistake}
			Before using an event it has to be cloned, because it may be part of an internal pool and may be reused by the framework. \\
		
	\section*{2. (Activity Lifecycle)}
		Resumed: onResume(); \\
		Paused: onPause(); \\
		Stopped: onStop(); \\
	
	\section*{3. (Resources)}
		Strings should be defined in the value directory, a subdirectory from resources (res), in a string.xml file. \\
		The advantage is to separate the model from the view (MVC). That means that the displayed text can be changed without changing the logic of the application (e.g.: change the language). \\
	
	\section*{4.(What are Intents?)}
		An intent is an abstract description of an operation to be performed. \\
		Intents are used to exchange information between operations (e.g.: activities, services). \\
		Explicit Intents specify the action and the components which should be used (e.g.: start an activity). Implicit Intents don't specify the Android components which should be called. The user may choose the Android component at runtime (e.g.: go to a specific website but the user can choose the browser). \\
	
	\section*{5. (Service Lifecycle)}
		\subsection*{a)}
			false, Context.stopService(); stops the service before it finished its job
		
		\subsection*{b)}
			true
		
		\subsection*{c)}
			true
		
		\subsection*{d)}
			false
	
	\section*{6. (Android Manifest)}
	
	
\end{document}